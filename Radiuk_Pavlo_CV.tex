% resume.tex
% vim:set ft=tex spell:

\documentclass[10pt,letterpaper]{article}
\usepackage[letterpaper,margin=0.75in]{geometry}
\usepackage[utf8]{inputenc}
\usepackage{mdwlist}
\usepackage[T1]{fontenc}
\usepackage{textcomp}
\usepackage{tgpagella}
\usepackage{latexsym}
\usepackage{amssymb}
\usepackage{hyperref}
\usepackage[T1,T2A]{fontenc}
\usepackage[english, russian]{babel}
\usepackage[babel=true]{microtype}
\usepackage{tempora}  % Times for numbers in math mode
\usepackage{newtxmath}  % Times in math mode
\pagestyle{empty}
\setlength{\tabcolsep}{0em}

% indentsection style, used for sections that aren't already in lists
% that need indentation to the level of all text in the document
\newenvironment{indentsection}[1]%
{\begin{list}{}%
	{\setlength{\leftmargin}{#1}}%
	\item[]%
}
{\end{list}}

% opposite of above; bump a section back toward the left margin
\newenvironment{unindentsection}[1]%
{\begin{list}{}%
	{\setlength{\leftmargin}{-0.5#1}}%
	\item[]%
}
{\end{list}}

% format two pieces of text, one left aligned and one right aligned
\newcommand{\headerrow}[2]
{\begin{tabular*}{\linewidth}{l@{\extracolsep{\fill}}r}
	#1 &
	#2 \\
\end{tabular*}}

% make "C++" look pretty when used in text by touching up the plus signs
\newcommand{\CPP}
{C\nolinebreak[4]\hspace{-.05em}\raisebox{.22ex}{\footnotesize\bf ++}}

% and the actual content starts here
\begin{document}

\begin{center}
{\LARGE \textbf{Pavlo Radiuk}}

%\url{} \textbullet
\ \ \texttt{radiukpavlo@gmail.com} \textbullet
\ \ +38 (097) 851 94 48 
\\
\ \ \href{https://www.linkedin.com/in/pavlo-radiuk-487029123/}{LinkedIn account} \textbullet
\ \ \href{https://github.com/soolstafir}{Github repository}
\\
Khmelnytsky, Ukraine
\end{center}


\hrule
\vspace{-0.4em}
\subsection*{Education}

\begin{itemize}
	\parskip=0.1em
	
	\item 
	\headerrow
		{\textbf{Khmelnytskyi National University}}
		{\textbf{Khmelnytsky, Ukraine}}
	\\
	\headerrow
		{\emph{Department of Applied Mathematics and Social 
Informatics, Ph.D. Applied Mathematics}}
		{\emph{09/2017 -- Present}}
	\begin{itemize*}
		\item For research purposes created several Python scripts to recognize images of diverse datasets using TensorFlow and MATLAB as a part of the doctoral thesis.
	\end{itemize*}
	
	\begin{itemize*}
		\item Thesis: \emph{Methods of optimization of architectural and training parameters of convolutional neural networks for classification tasks of medical diagnostics static images}
	\end{itemize*}
	
	\item 
	\headerrow
		{\textbf{Khmelnytskyi National University}}
		{\textbf{Khmelnytsky, Ukraine}}
	\\
	\headerrow
		{\emph{Department of Applied Mathematics and Social 
Informatics, Master of Science \footnote{\url{https://drive.google.com/open?id=1a10K3d_8TxrRu7NT-zE5a6jSWdQLITic}}}}
		{\emph{09/2011 -- 01/2017}}
	
	\begin{itemize*}
		\item Had to deal with tracking cars in MATLAB environment. Developed an application for selecting the optimal parameters for the tracking system based on collected by myself statistics.
	\end{itemize*}
	
	\begin{itemize*}
		\item Worked on the creation of a system to produce a linear regression model and its research. Task was to create a class on C\# in Visual Studio. The class collected results of major operations on matrices, such as multiplication of matrices, transpose matrices, inverse matrices, etc.
	\end{itemize*}
	
	\begin{itemize*}
		\item Obtained master's degree in the field of Mathematical and Computer Modeling and qualification of Mathematician-analyst, the teacher of a higher educational institution.
	\end{itemize*}
	
	\begin{itemize*}
		\item Thesis: \emph{Optimization of filter's size distribution in convolutional neural networks for classification problems}
	\end{itemize*}

\end{itemize}

\hrule
\vspace{-0.4em}
\subsection*{Experience}

\begin{itemize}
	\parskip=0.1em
	
	\item
	\headerrow
		{\textbf{Khmelnytskyi National University}}
		{\textbf{Khmelnytsky, Ukraine}}
	\\
	\headerrow
		{\emph{Senior Laboratory Assistant}}
		{\emph{12/2016 -- 08/2017}}
	\begin{itemize*}
		\item Provided the maintenance of computer equipment, software, and a local computer lab network.
		\item Carried out the administration both of workplaces and users of the computer lab.
		\item Prepared the materials, control instruments, and multimedia, that are necessary for the implementation of the educational process.
	\end{itemize*}

\end{itemize}

%\hrule
%\vspace{-0.4em}
%\subsection*{Awards}
%
%\begin{itemize}
%	\parskip=0.1em
%	
%	\item 
%	\headerrow
%		{Scholarship of the Irish Research Council}
%		{\emph{10/2015 -- Present}}
%	\item 
%	\headerrow
%		{\emph{Cusanuswerk} scholarship of the German state}
%		{\emph{04/2014 -- 09/2015}}	
%	\item 
%	\headerrow
%		{Microsoft Certified Professional (Programming in C%\#)}
%		{\emph{06/2015}}
%	\item 
%	\headerrow
%		{Best Delegate award in various Model United Nations conferences}
%		{\emph{11/2012 -- 01/2014}}
%	\item 
%	\headerrow
%		{Second and third prizes \emph{Bundeswettbewerb Fremdsprachen}, national foreign languages competition}
%		{\emph{2007 -- 2008}}
%	\item 
%	\headerrow
%		{First and second prizes \emph{Landeswettbewerb Mathematik}, state mathematics competition}
%		{\emph{2006 -- 2008}}	
%
%\end{itemize}

\hrule
\vspace{-0.4em}
\subsection*{Languages and Technologies}

\begin{indentsection}{\parindent}
\hyphenpenalty=1000
\begin{description*}
	\item[Programming Languages:]
	Python, C\# (academic level), SQL, \LaTeX, MATLAB
	\item[Technologies:]
	SciPy, NumPy, Keras, TensorFlow, Google Analytics, Git, Docker
	\item[Languages:]
	Fluent in Ukrainian and Russian, advanced in English
	\item[Open Source Contributions:]
	The OpenCog Foundation
\end{description*}
\end{indentsection}

\hrule
\vspace{-0.4em}
\subsection*{Publications}

\begin{enumerate}
	\parskip=0.1em
	
	\item Радюк П. М. Реалізація нейромережевого алгоритму пакета інструментів MatConvNet з використанням графічного процесора [Текст] / П. М. Радюк, В. В. Романюк, Н. В. Грипинська. // Сборник трудов X Международной научной конференции "Наука и образование". – 2017. – №10. – С. 60–65.
	
	\item Radiuk, P. \href{https://doi.org/10.1515/itms-2017-0003}{Impact of Training Set Batch Size on the Performance of Convolutional Neural Networks for Diverse Datasets}. Information Technology and Management Science, 20(1), pp. 20-24. Retrieved 14 Jan. 2018. https://doi.org/10.1515/itms-2017-0003

\end{enumerate}

\end{document}

